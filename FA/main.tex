\documentclass[russian,]{article}
\usepackage[]{amsmath}
\usepackage{amssymb,amsmath}
\usepackage{ifxetex,ifluatex}
\usepackage{fixltx2e} % provides \textsubscript
\ifnum 0\ifxetex 1\fi\ifluatex 1\fi=0 % if pdftex
  \usepackage[T1, T2A]{fontenc}
  \usepackage[utf8]{inputenc}
\else % if luatex or xelatex
  \ifxetex
    \usepackage{mathspec}
  \else
    \usepackage{fontspec}
  \fi
  \defaultfontfeatures{Ligatures=TeX,Scale=MatchLowercase}
\fi
% use upquote if available, for straight quotes in verbatim environments
\IfFileExists{upquote.sty}{\usepackage{upquote}}{}
% use microtype if available
\IfFileExists{microtype.sty}{%
\usepackage[]{microtype}
\UseMicrotypeSet[protrusion]{basicmath} % disable protrusion for tt fonts
}{}
\PassOptionsToPackage{hyphens}{url} % url is loaded by hyperref
\usepackage[unicode=true]{hyperref}
\hypersetup{
            pdfborder={0 0 0},
            breaklinks=true}
\urlstyle{same}  % don't use monospace font for urls
\ifnum 0\ifxetex 1\fi\ifluatex 1\fi=0 % if pdftex
  \usepackage[shorthands=off,main=russian]{babel}
\else
  \usepackage{polyglossia}
  \setmainlanguage[]{}
\fi
\usepackage{longtable,booktabs}
% Fix footnotes in tables (requires footnote package)
\IfFileExists{footnote.sty}{\usepackage{footnote}\makesavenoteenv{long table}}{}
\usepackage{graphicx,grffile}
\makeatletter
\def\maxwidth{\ifdim\Gin@nat@width>\linewidth\linewidth\else\Gin@nat@width\fi}
\def\maxheight{\ifdim\Gin@nat@height>\textheight\textheight\else\Gin@nat@height\fi}
\makeatother
% Scale images if necessary, so that they will not overflow the page
% margins by default, and it is still possible to overwrite the defaults
% using explicit options in \includegraphics[width, height, ...]{}
\setkeys{Gin}{width=\maxwidth,height=\maxheight,keepaspectratio}
\IfFileExists{parskip.sty}{%
\usepackage{parskip}
}{% else
\setlength{\parindent}{0pt}
\setlength{\parskip}{6pt plus 2pt minus 1pt}
}
\setlength{\emergencystretch}{3em}  % prevent overfull lines
\providecommand{\tightlist}{%
  \setlength{\itemsep}{0pt}\setlength{\parskip}{0pt}}
\setcounter{secnumdepth}{0}
% Redefines (sub)paragraphs to behave more like sections
\ifx\paragraph\undefined\else
\let\oldparagraph\paragraph
\renewcommand{\paragraph}[1]{\oldparagraph{#1}\mbox{}}
\fi
\ifx\subparagraph\undefined\else
\let\oldsubparagraph\subparagraph
\renewcommand{\subparagraph}[1]{\oldsubparagraph{#1}\mbox{}}
\fi

% set default figure placement to htbp
\makeatletter
\def\fps@figure{htbp}
\makeatother

\usepackage[a4paper,left=20mm,right=10mm,top=20mm,bottom=20mm]{geometry}

\usepackage{amsgen, amsmath, amstext, amsbsy, amsopn, amsfonts, amsthm, amssymb, amscd, mathtext, mathtools}
\usepackage{versions}

\usepackage{float}
\restylefloat{table}

\usepackage{xfrac}

\usepackage{hyperref,texlinks}

\usepackage{minted}
\newcommand{\mi}{\mintinline}

% Set line spacing for the whole document
\renewcommand{\baselinestretch}{1}

% RedeclareMathOperator
\makeatletter
\newcommand\RedeclareMathOperator{%
  \@ifstar{\def\rmo@s{m}\rmo@redeclare}{\def\rmo@s{o}\rmo@redeclare}%
}
% this is taken from \renew@command
\newcommand\rmo@redeclare[2]{%
  \begingroup \escapechar\m@ne\xdef\@gtempa{{\string#1}}\endgroup
  \expandafter\@ifundefined\@gtempa
     {\@latex@error{\noexpand#1undefined}\@ehc}%
     \relax
  \expandafter\rmo@declmathop\rmo@s{#1}{#2}}
% This is just \@declmathop without \@ifdefinable
\newcommand\rmo@declmathop[3]{%
  \DeclareRobustCommand{#2}{\qopname\newmcodes@#1{#3}}%
}
\@onlypreamble\RedeclareMathOperator
\makeatother

% Explain
\newcommand{\expl}[2]{\underset{\mathclap{\overset{\uparrow}{#2}}}{#1}}
\newcommand{\explup}[2]{\overset{\mathclap{\underset{\downarrow}{#2}}}{#1}}
\newcommand{\obrace}[2]{\overbrace{#1}^{#2}}
\newcommand{\ubrace}[2]{\underbrace{#1}_{#2}}

% Arrows
\newcommand{\Then}{\Rightarrow}
\newcommand{\Iff}{\Leftrightarrow}
\newcommand{\When}{\Leftarrow}
\newcommand{\Bydef}{\rightleftharpoons}
%\newcommand{\Divby}{\raisebox{-2pt}{\vdots}}
\DeclareRobustCommand{\Divby}{%
  \mathrel{\text{\vbox{\baselineskip.65ex\lineskiplimit0pt\hbox{.}\hbox{.}\hbox{.}}}}%
}

\DeclareMathOperator{\Char}{char}
\DeclareMathOperator{\Ker}{Ker}
\DeclareMathOperator{\Quot}{Quot}
\DeclareMathOperator{\Gal}{Gal}
\DeclareMathOperator{\Aut}{Aut}
\DeclareMathOperator{\id}{id}
\RedeclareMathOperator{\Im}{Im}
\DeclareMathOperator{\ord}{ord}

% Mathbbs
\newcommand{\N}{\mathbb{N}}
\newcommand{\Z}{\mathbb{Z}}
\newcommand{\Zp}{\Z_p}
\newcommand{\Q}{\mathbb{Q}}
\newcommand{\R}{\mathbb{R}}
\renewcommand{\C}{\mathbb{C}}
\newcommand{\F}{\mathbb{F}}

% Short names
\renewcommand{\~}{\sim}
\renewcommand{\phi}{\varphi}
\newcommand{\Have}{\hookrightarrow}
\newcommand{\isom}{\cong}
\renewcommand{\ge}{\geqslant}
\renewcommand{\le}{\leqslant}
\providecommand{\abs}[1]{\lvert#1\rvert}
\renewcommand{\bf}{\textbf}
\renewcommand{\it}{\textit}
\renewcommand{\i}{\item}
\newcommand{\ul}{\itemize}
\renewcommand{\bar}{\overline}
%\newcommand{\ol}{\enumerate}
%\let\ol\enumerate
\newenvironment{ol}{\begin{enumerate}}{\end{enumerate}}
\renewcommand{\b}{\begin}
\newcommand{\e}{\end}
\newcommand{\<}{\langle}
\renewcommand{\>}{\rangle} % Возможно сломает форматирование

\usepackage{pifont}
\newcommand{\cmark}{\ding{51}}
\newcommand{\xmark}{\ding{55}}
\newcommand{\y}{\cmark}
\newcommand{\x}{\xmark}

% Quotes
\usepackage{xspace}
\newcommand{\lgq}{\guillemotleft\nobreak\ignorespaces}
\newcommand{\rgq}{\guillemotright\xspace}

% Consider changing to sfrac
\newcommand{\bigslant}[2]{{\raisebox{.2em}{$#1$}\left/\raisebox{-.2em}{$#2$}\right.}}

\makeatletter
\newenvironment{sqcases}{%
  \matrix@check\sqcases\env@sqcases
}{%
  \endarray\right.%
}
\def\env@sqcases{%
  \let\@ifnextchar\new@ifnextchar
  \left\lbrack
  \def\arraystretch{1.2}%
  \array{@{}l@{\quad}l@{}}%
}
\makeatother

\makeatletter
\newenvironment{nocases}{%
  \matrix@check\sqcases\env@sqcases
}{%
  \endarray\right.%
}
\def\env@nocases{%
  \let\@ifnextchar\new@ifnextchar
  \def\arraystretch{1.2}%
  \array{@{}l@{\quad}l@{}}%
}
\makeatother

% hide LaTeX code from pandoc
\newcommand{\nopandoc}[1]{#1}
\nopandoc{
	\let\Begin\begin
	\let\End\end
}


% Styles for theorems
\usepackage{thmtools}
% renewtheorem
% not works
\makeatletter
\def\renewtheorem#1{%
  \expandafter\let\csname#1\endcsname\relax
  \expandafter\let\csname c@#1\endcsname\relax
  \gdef\renewtheorem@envname{#1}
  \renewtheorem@secpar
}
\def\renewtheorem@secpar{\@ifnextchar[{\renewtheorem@numberedlike}{\renewtheorem@nonumberedlike}}
\def\renewtheorem@numberedlike[#1]#2{\newtheorem{\renewtheorem@envname}[#1]{#2}}
\def\renewtheorem@nonumberedlike#1{  
\def\renewtheorem@caption{#1}
\edef\renewtheorem@nowithin{\noexpand\newtheorem{\renewtheorem@envname}{\renewtheorem@caption}}
\renewtheorem@thirdpar
}
\def\renewtheorem@thirdpar{\@ifnextchar[{\renewtheorem@within}{\renewtheorem@nowithin}}
\def\renewtheorem@within[#1]{\renewtheorem@nowithin[#1]}
\makeatother

\declaretheoremstyle[notefont=\bfseries,notebraces={}{},headpunct={},%
postheadspace={5px},headformat={\makebox[0pt][r]{\NAME\ \NUMBER\ }\setbox0\hbox{\ }\hspace{-\the\wd0}\NOTE}]{problemstyle}
\declaretheorem[style=problemstyle,numbered=no,name=№]{problem}

\declaretheoremstyle[notefont=\bfseries,notebraces={}{},headpunct={ },postheadspace={0px},qed=$\blacktriangleleft$,
headformat={\makebox[0pt][r]{\NAME\ }\setbox0\hbox{\ }\hspace{-\the\wd0}\NOTE},]{solutionstyle}
\declaretheorem[style=solutionstyle,numbered=no,name=$\blacktriangleright$]{solution}
\let\proof\relax
\declaretheorem[style=solutionstyle,numbered=no,name=$\blacktriangleright$]{proof}

\declaretheoremstyle[notefont=\bfseries,notebraces={}{},headpunct={.},postheadspace={4px}]{definitionstyle}
\declaretheorem[style=definitionstyle,numbered=yes,name=Опр]{defn}
\declaretheorem[style=definitionstyle,numbered=no,name=Утв]{stmt}
\declaretheorem[style=definitionstyle,numbered=no,name=Зам]{note}
\declaretheorem[style=definitionstyle,numbered=no,name=Теор]{thm}
\let\lemma\relax
\declaretheorem[style=definitionstyle,numbered=no,name=Лемма]{lemma}

\usepackage{enumitem}
\AddEnumerateCounter{\Asbuk}{\@Asbuk}{\CYRM}
\AddEnumerateCounter{\asbuk}{\@asbuk}{\cyrm}
%\renewcommand{\theenumi}{(\Asbuk{enumi})}
%\renewcommand{\labelenumi}{\Asbuk{enumi})}
\setlist[itemize]{leftmargin=*}
\setlist[enumerate]{leftmargin=*}

\usepackage[at]{easylist}

\def\definemyeasylist#1#2{\expandafter\def\csname el@style@#1\endcsname{\NewList(#2)}}
\def\el{\futurelet\next\domyeasylist}
\def\domyeasylist{\ifx[\next\expandafter\domyeasylistone\else\expandafter\domyeasylistnop\fi}
\def\domyeasylistone[#1]{\begin{easylist}\if\relax\detokenize{#1}\relax\else\csname el@style@#1\endcsname\fi}
\def\domyeasylistnop{\begin{easylist}\NewList}
\def\endel{\end{easylist}}

\definemyeasylist{ul}{Margin1=0cm,Progressive*=1em,
Hang=true,Space=-1pt,Space*=-1pt,Hide=1000,%
Style1*=\textbullet\hskip .5em,%
Style2*=--\hskip .5em,%
Style3*=$\ast$\hskip .5em,%
Style4*=$\cdot$\hskip .5em}

\definemyeasylist{ol}{Hang=true,Mark=.,Space=-1pt,Space*=-1pt,Align=move,
Style2*={(},Mark2={)},Numbers2=l,Hide2=1,%
Numbers3=r,Hide3=2,%
Numbers4=L,Hide4=3}

% Strikeout
\usepackage[normalem]{ulem}
\newcommand{\s}[1]{\ifmmode\text{\sout{\ensuremath{#1}}}\else\sout{#1}\fi}

% Tables
\usepackage{color}
\usepackage{colortbl}
\usepackage{pbox}

\usepackage{tikz}
\usetikzlibrary{chains,shapes,arrows,positioning}



\usepackage{tikz}

\newcommand{\ra}{\rightarrow}
\newcommand{\cross}{\times}
\newcommand{\bf}{\textbf}
\newcommand{\eps}{\varepsilon}
\newcommand{\norm}[1]{\left\lVert#1\right\rVert}
\newcommand{\wave}{\tilde}

\begin{document}

\section{МП и ТП}

\begin{defn}
Пусть $X$ --- некоторое мн-во. Функция $\rho:X \cross X \ra \R^+$ наз-ся \textbf{метрикой} в мн-ве $X$, если:
\begin{enumerate}
	\i $\rho(x, y) \ge 0 \forall x, y \in X$ и $\rho(x, y) = 0 \Iff x=y$
	\i $\rho(x, y) = \rho(y, x) \forall x, y \in X$
	\i $\rho(x, y) \le \rho(x, z)+\rho(z, y) \forall x, y, z \in X$
\end{enumerate}
Множество $X$ с введённой на нём метрикой $\rho$ называется \bf{метрическим пространством} $(X, \rho)$.
\end{defn}

\begin{defn}
Пр-во $Y$ наз-ся \bf{подпространством} пр-ва $X$, если $Y \subset X$, и на $Y$ берётся \bf{индуцированная метрика} $\rho_Y(y_1, y_2) = \rho_X(y_1, y_2) \forall y_1, y_2 \in Y$. 
\end{defn}

\begin{defn}
\bf{Расстояние между множествами} есть $\rho(A, B) = \inf\limits_{a \in A, b \in B} \rho(a, b)$
\end{defn}

\begin{defn}
Мн-во $S$ из метр. пр-ва $(X, \rho)$ наз-ся \bf{всюду плотным} в $X$, если его замыкание совпадает с $X$, т. е. $\forall x \in X \forall \eps > 0 \exists y \in S: \rho(x, y)<\eps$.	
\end{defn}

\begin{defn}
МП $(X, \rho)$ наз-ся \bf{сепарабельным}, если в нём существует счётное всюду плотное мн-во. 
\end{defn}

\begin{defn}
Пусть $X$ --- некоторое сем-во множеств. Семейство $\tau$ подмножеств мн-ва $X$ наз-ся \bf{топологией}, если:
\begin{enumerate}
\i $X \in \tau$ и $\emptyset \in \tau$
\i $\forall$ семейства подмн-в $\{U_\alpha \mid \alpha \in A\} \subset \tau \Have \bigcup\limits_{\alpha \in A} U_\alpha \in \tau$
\i $\forall$ конечного семейства подмн-в ${U_k \mid k \in \bar{1,N}} \subset \tau \Have \bigcup\limits_{k=1}^N U_k \in \tau$
\end{enumerate}
Мн-во $X$ со введённой на нём топологией $\tau$ наз-ся \bf{топологическим пр-вом (ТП)} $(X, \tau)$.
\end{defn}

\begin{defn}
Пусть $(X, \tau)$ --- ТП. Любое мн-во $U \in \tau$ наз-ся \bf{открытым}($\tau$-открытым) в ТП $(X, \tau)$. Топология $\tau$ называется \bf{семейством открытых подмн-в} мн-ва $X$.
\end{defn}

\begin{defn}
Пусть $(X, \tau)$ --- ТП. Для любого $x \in X$ \bf{окрестностью} $x$ называется произвольное $\tau$-открытое множество, содержащее $x$.
\end{defn}

\begin{defn}
Пусть $(X, \tau)$ --- ТП, $S \subset X$. \bf{Открытым покрытием} мн-ва $S$ наз-ся сем-во $\tau$-открытых мн-в $\{U_\alpha \mid \alpha \in A\}$, т. ч. $S \subset \bigcup\limits_{\alpha \in A} U_\alpha$.
\end{defn}

\section{ПМП}

\begin{defn}
П-ть $\{x_n\}_{n=1}^{\infty}$ элементов МП $(X, \rho)$ наз-ся \bf{фундаментальной} (или $\rho$-фундаментальной), если $\forall \eps > 0 \exists N = N(\eps):\forall n, m > N \Have \rho(x_n, x_m) < \eps$.
\end{defn}

\begin{defn}
МП $(X, \rho)$ наз-ся \bf{полным}, если любая фундаментальная п-ть из $(X, \rho)$ сх-ся.
\end{defn}

\begin{defn}
МП $(X, \rho)$ наз-ся \bf{связным}, если $X$ нельзя представить в виде объединения двух непустых непересекающихся открытых множеств.
\end{defn}

\begin{defn}
Пусть $(X, \rho)$ --- МП. \bf{Открытым шаром} с центром в точке $x \in X$ радиуса $R>0$ наз-ся мн-во $B_R(x) = B(x, R) = \{y \in X \mid \rho(x, y) < R\}$.
\bf{Замкнутым шаром} с центром в точке $x \in X$ радиуса $R>0$ наз-ся мн-во $\bar{B}_R(x) = \bar{B}(x, R) = \{y \in X \mid \rho(x, y) \le R\}$.
\end{defn}

\begin{defn}
Мн-во $M \subset X$, где $(X, \rho)$ --- МП, наз-ся \bf{открытым}, если $\forall x_0 \in M \exists \eps > 0: B_\eps(x_0) \subset M$. Обычно обозначается $G$.
\end{defn}

\begin{defn}
Пусть $(X, \rho)$ --- МП и $M \subset X$. Точка $x_0 \in X$ наз-ся \bf{точкой прикосновения} мн-ва $M$, если $\forall \epsilon > 0 \Have B_\eps(x_0) \cap M \ne \emptyset$.
\end{defn}

\begin{defn}
Пусть $(X, \rho)$ --- МП и $M \subset X$. Точка $x_0 \in X$ наз-ся \bf{предельной точкой} мн-ва $M$, если $\exists y \in B_eps(x_0)\cap M: y \ne x$.
\end{defn}

\begin{defn}
Пусть $(X, \rho)$ --- МП и $M \subset X$. Множество $[M]$(или $\bar{M})$ наз-ся \bf{замыканием} мн-ва $M$, если оно получено добавлением к $M$ всех его точек прикосновения.
\end{defn}

\begin{defn}
Мн-во $M \subset X$, где $(X, \rho)$ --- МП, наз-ся \bf{замкнутым}, если $[M] = M$. Обычно обозначается $F$.
\end{defn}

\begin{defn}
Лебегово пр-во $l_p$ для $1 \le p < +\infty$ состоит из числовых п-тей вида $x = (x_1, x_2, \dots)$

$l_p = \{x: \N \ra \R \mid \sum\limits_{k=1}^{\infty} |x(k)|^p <+\infty\}$ 

с нормой $\norm{x}_p = \sqrt[p]{\sum\limits_{k=1}^{\infty} |x(k)|^p}$ и метрикой $\rho_p(x, y) = \norm{x-y}_p$.

\bf{Полное сепарабельное}
\end{defn}

\begin{defn}
Лебегово пр-во $l_\infty$. 
С нормой $\norm{x}_\infty = \sup\limits_{k \in \N} |x(k)|$ и метрикой $\rho_\infty(x, y) = \norm{x-y}_\infty$.
\bf{Полное несепарабельное}
\end{defn}

\begin{defn}
Пр-во $C[a,b] = \{x: [a, b] \ra \R \mid x\text{ непрерывно на } [a, b]\}$ непрерывных на $[a, b]$ функций с нормой $\norm{x}_C = \sup\limits_{t \in [a, b]} |x(t)|$ и метрикой $\rho_C(x, y) = \norm{x-y}_C$.

\bf{Полное сепарабельное связное}
\end{defn}

\begin{thm}[Достаточное условие несепарабельности МП]
Пусть в МП $(X, \rho)$ существует несчётное подмножество $A_0$ и $\exists \eps_0 > 0: \forall a, b \in A_0, a \ne b, \Have \rho(a, b) \ge \eps_0$. Тогда МП $(X, \rho)$ является несепарабельным.
\end{thm}

\begin{thm}[2.1, Принцип вложенных шаров]
Пусть $X$ --- ПМП, $\{B_n := \bar{B}(x_n, r_n)\}$ --- п-ть замкн. вложенных шаров, $r_n \ra 0$. Тогда $\exists! x \in \bigcap\limits_n B_n$.
\end{thm}

\begin{thm}[2.2, Бэр]
Пусть $X$ --- ПМП, тогда $X$ нельзя представить в виде $X = \bigcup\limits_{n=1}^\infty M_n$, где $M_n$ --- нигде не плотное множество.
\end{thm}

\begin{defn}
Пусть $(X, \rho)$ --- МП. Отображение $f: X \ra X$ наз-ся \bf{сжимающим}, если $\exists \alpha \in (0, 1): \forall x, y \in X \Have \rho(f(x), f(y)) \le \alpha \rho(x, y)$.
\end{defn}

\begin{defn}
Пусть X --- некоторое мн-во, $f: X \ra X$ --- отображение. Точка $x_o \in X$ наз-ся \bf{неподвижной} для отображения $f$, если $f(x_0) = x_0$.
\end{defn}

\begin{thm}[2.3, Банаха о сжимающих отображениях]
Пусть $(X, \rho)$ --- ПМП, $f: X \ra X$ --- сжимающее отображение. Тогда $f$ имеет единственную неподвижную точку.
\end{thm}

\begin{defn}
Пусть $(X_1, \rho_1)$ и $(X_2, \rho_2)$ --- МП. Отображение $\phi: X_1 \ra X_2$ называется \bf{изометрией}, если $\phi$ --- биекция и $\forall x_1, y_1 \in X_1 \Have \rho_1(x_1, y_1) = \rho_2(\phi(x_1), \phi(y_1))$. Если между МП $X_1$ и $X_2$ существует изометрия, они называются \bf{изометричными}.
\end{defn}

\begin{defn}
ПМП $(Y, d)$ наз-ся \bf{пополнением} МП $(X, \rho)$, если $\exists Z \subset Y$, $Z$ --- всюду плотное в $Y$, т. ч. МП $(X, \rho)$ и $(Z, d)$ изометричны.
\end{defn}

\begin{thm}[2.4, Хаусдорф]
Пусть $X$ --- неполное МП, тогда $\exists$ ПМП $Y$ --- пополнение $X$
\end{thm}

\begin{defn}
Пусть $(X, \rho)$ --- МП. Мн-во $A \subset X$ наз-ся \bf{плотным в мн-ве $B \subset X$}, если $B \subset \bar{A}$, т. е. $\forall b \in B \forall \eps > 0 \exists a \in A: \rho(a, b) < \eps$.

$A$ наз-ся \bf{всюду плотным}, если $A$ плотно в $X$.

$A$ \bf{нигде не плотно}, если оно не плотно ни в одном шаре, т. е. в каждом шаре $B \subset R$ содержится другой шар $B', B' \cap A = \emptyset$.
\end{defn}

\section{КМП и КТП}

\begin{defn}
Пусть $(X, \tau)$ --- ТП. Мн-во $S \subset X$ наз-ся \bf{компактным}, если любое открытое покрытие мн-ва $S$ содержит конечное подпокрытие
%\begin{enumerate}
%\i \bf{компактным (бикомпактным)}, если любое открытое покрытие мн-ва $S$ содержит конечное подпокрытие
%\i \bf{счётно-компактным}, если любое счётное открытое покрытие мн-ва $S$ содержит конечное подпокрытие
%\i \bf{секвенциально компактным}, если любая п-пть эл-тов мн-ва $S$ содержит п-ть, сходящуюся по топологии $\tau$ к некоторому эл-ту $S$
%\end{enumerate}
\end{defn}

\begin{defn}
Пусть $(X, \tau)$ --- ТП. Говорят, что п-ть $\{x_n\}_{n=1}^\infty \subset X$ \bf{сх-ся по топологии $\tau$} к элементу $x \in X$, если $\forall U(x) \exists N: \forall n > N \Have x_n \in U(x)$. Обозначается $x_n \ra_{n \ra \infty}^\tau x$.
\end{defn}

%\begin{thm}
%Бикомпактность 
%Счётная компактность
%Секвенциальная компактность
%\end{thm}

\begin{defn}
Пусть $X$ --- ТП. $\{B_\alpha\}$ наз-ся \bf{центрированной системой множеств (ЦС)}, если любая их конечная подсистема имеет непустое пересечение.
\end{defn}

\begin{thm}[3.1, Критерий компактности топологических пространств]
Пусть $X$ --- ТП. Тогда $X$ --- компакт $\Iff$ в $X$ любая ЦС замкнутых подмножеств имеет непустое пересечение.
\end{thm}

\begin{defn}
Пусть $(X, \rho)$ --- МП. Мн-во $S \subset X$ наз-ся \bf{вполне ограниченным}, если $\forall \eps > 0 \exists$ конечный набор точек $x_1, \dots, x_N \in S: S \subset \bigcup\limits_{k=1}^N B_\eps(x_k)$. Указанный набор точек называется \bf{конечной $\eps$-сетью} мн-ва S.
\end{defn}

\begin{thm}[3.2, Критерий компактности метрических пространств]
Пусть $X$ --- МП. Тогда СУЭ:
\begin{enumerate}
\i $X$ --- компакт
\i МП $X$ --- полное и ВО
\i мн-во $X$ явл. секвенциально компактным (т. е. из любой п-ти $\{x_n\} \subset X$ можно выделить сходящуюся п-пть
\i любое бесконечное множество в $X$ имеет предельную точку
\end{enumerate} 
\end{thm}
\begin{cor}
Если $X$ --- МП, то $M \subset X$ --- компактно $\Then$ $M$ --- замкн.
\end{cor}
\begin{cor}
Если $X$ --- ПМП, то $M \subset X$ --- компактно $\Iff$ $M$ --- замкн. и ВО.
\end{cor}
\begin{cor}
Если $X = \R^n$, то $M \subset X$ --- компактно $\Iff$ $M$ --- замкн. и огр.
\end{cor}

\begin{defn}
Пусть $(X, \rho)$ --- МП. Мн-во $M \subset X$ наз-ся \bf{ограниченным}, если $\exists x_0 \in X: \exists r > 0: M \subset B_r(x_0)$.
\end{defn}

\begin{defn}
Мн-во метрического пространства наз-ся \bf{предкомпактом}, если его замыкание компактно.
\end{defn}

\begin{defn}
Сем-во функций $S \subset C[a, b]$ наз-ся \bf{равномерно ограниченным}, если $\exists c: \forall f \in S \max\limits_{x \in [a, b]} |f(x)| \le c$.
\end{defn}

\begin{defn}
Сем-во функций $S \subset C[a, b]$ наз-ся \bf{равностепенно непрерывным}, если $\forall \eps > 0 \exists \delta > 0: \forall f \in S \forall x, x' \in [a, b] \Have |x-x'| < \delta \Then |f(x)-f(x')| < \eps$.
\end{defn}

\begin{thm}[3.3, Теорема Арцела-Асколи]
Пусть $X$ --- КМП. Тогда сем-во ф-ий $S \subset C(X)$ предкомпактно в пр-ве $C(X)$ $\Iff$ оно равномерно ограниченно и равностепенно непрерывно.
\end{thm}

\begin{defn}
Пр-во $C^k[a, b]$ k раз непр. дифф. ф-ий $x:[a, b] \ra \R$ с нормой $\norm{x}_{C^k} = \sum\limits_{i=0}^k \max\limits_{x \in [a, b]} |x^{(i)}(t)|$.
\end{defn}

\section{ЛНП}

\begin{defn}
Непустое мн-во $L$ наз-ся \bf{линейным} (или \bf{векторным}) пр-вом над $M$, если:
$\forall x, y, z \in L \forall \alpha, \beta \in M$:
\begin{enumerate}
\i однозначно определён элемент $x+y \in L$:
 \begin{enumerate}
 \i $x+y = y+x$
 \i $x+(y+z) = (x+y)+z$
 \i $\exists 0 \in L: x+0 = 0+x = x$
 \i $\exists (-x) \in L: x+(-x) = 0$
 \end{enumerate}
\i однозначно определён элемент $\alpha x \in L$:
 \begin{enumerate}
 \i $\alpha(\beta x) = (\alpha \beta)x$
 \i $\exists 1 \in L: 1x=x$
 \i $(\alpha+\beta)x = \alpha x + \beta x$
 \i $\alpha(x+y) = \alpha x + \alpha y$
 \end{enumerate}
\end{enumerate}
В зависимости от $M$ различают действительные ($M=\R$) и комплексные ($M=\C$) линейные нормированные пространства.
\end{defn}

\begin{defn}
Пусть $X$ --- комплексное ЛП. Ф-ия $\norm{\cdot}: X \ra \R$ наз-ся \bf{нормой} в $X$, если:
\begin{enumerate}
\i $\forall x \in X \Have \norm{x} \ge 0, \norm{x} = 0 \Iff x = 0$
\i $\forall x \in X \forall t \in \R \norm{tx} = |t| \norm{x}$
\i $\forall x, y \in X \Have \norm{x+y} \le \norm{x} + \norm{y}$ (нер-во треугольника)
\end{enumerate}
Любое пр-во с фиксированной в нём нормой будем называть \bf{линейным нормированным пространством (ЛНП)}.
\end{defn}

\begin{defn}
$\norm{\cdot}_1$ и $\norm{\cdot}_2$ на ЛП $E$ наз. \bf{эквивалентными}, если $\exists c_1, c_2 \ge 0: \forall x \in E \Have c_1 \norm{x}_2 \le \norm{x}_1 \le c_2 \norm{x}_2$.
\end{defn}

\begin{defn}
Пусть $\norm{\cdot}_1$ и $\norm{\cdot}_2$ --- нормы на ЛП $E$. $\norm{\cdot}_1$ \bf{слабее} $\norm{\cdot}_2$, если $x_n \xrightarrow[\norm{\cdot}_2]{} x \Then x_n \xrightarrow[\norm{\cdot}_1]{} x$.
\end{defn}

\begin{thm}[4.1, Рисс]
Пусть $E$ --- НП, $\dim E = \infty$. Тогда $S(0, 1)$ --- единичная сфера --- не является компактной (даже не является ВО).
\end{thm}

\begin{defn}
Полное НП наз-ся \bf{банаховым (БП)}(обычно обозначается $B$) .
\end{defn}

\begin{defn}
ЛП наз-ся \bf{замкнутым}, если оно содержит все свои предельные точки.
\end{defn}

\begin{defn}
ЛП $L'$ наз-ся \bf{подпространством} ЛП $L$, если $L' \subset L$ и операции сложения векторов и умножения вектора на число определены так же, как в $L$.
\end{defn}

\begin{defn}
\bf{Линейной комбинацией (ЛК)} в-ров $x_1, \dots, x_n$ наз-ся любой в-р вида $\alpha_1 x_1 + \dots + \alpha_n x_n$, где $\alpha_1, \dots, \alpha_n$ --- числовые множители.
\end{defn}

\begin{defn}
ЛК наз. \bf{нетривиальной}, если хотя бы один из коэф. $\alpha_1, \dots, \alpha_n$ отличен от нуля.
\end{defn}

\begin{defn}
В-ры $x_1, \dots, x_n$ наз-ся \bf{линейно зависимыми (ЛЗ)}, если $\exists$ нетрив. ЛК, равная 0. Иначе они называются \bf{линейно независимыми (ЛНЗ)}.
\end{defn}

\begin{defn}
ЛП наз-ся \bf{n-мерным}, если в нём $\exists$ n ЛНЗ в-ров, а любые $n+1$ в-ров ЛЗ. В таком случае эти $n$ в-ров наз-ся базисом.
\end{defn}

\begin{defn}
ЛП наз-ся \bf{бесконечномерным}, если $\forall n \in \N$ в нём $\exists$ $n$ ЛНЗ в-ров.
\end{defn}

\begin{defn}
Пусть задана некоторая система эл-тов ЛП $L$. Совокупность всех ЛК этой системы наз-ся её \bf{линейной оболочкой}.
\end{defn}

\begin{defn}
Система эл-тов $\{x_\alpha, \alpha \in A\}$ наз-ся \bf{полной} в пр-ве $X$, если её ЛО плотна в $X$, т. е. если $\forall x \in X \forall \eps > 0 \exists \{x_{\alpha_1}, \dots, x_{\alpha_n} \subset \{x_\alpha, \alpha \in A\} \exists \lambda_1, \dots, \lambda_n: \norm{x-\sum\limits_{k=1}^n \lambda_k x_{\alpha_k}} < \eps$.
\end{defn}

\begin{defn}
П-ть эл-тов $e_1, e_2, \dots$ ЛНП $X$ наз-ся \bf{базисом} пр-ва $X$, если каэдый эл-т $x \in X$ имеет единственное разложение по этой системе, т. е. $\exists ! \{\lambda_n\}_{n=1}^\infty: x = \sum\limits_{n=1}^\infty \lambda_n e_n$. Здесь ряд сх-ся к эл-ту $x$ по норме пр-ва $X$, т. е. $\forall \eps > 0 \exists N \in \N: \forall n \ge N \Have \norm{x - \sum\limits_{k=1}^n \lambda_k x_{\alpha_k}} < \eps$.
\end{defn}

\begin{defn}
Пр-во $c$ сходящихся п-тей $x=(x_1, x_2, \dots)$ с операциями сложения и умножения на число и нормой $\norm{x}_C = \sup\limits_{k \in N} |x_k|$.
\end{defn}

\begin{defn}
Пр-во $c_0$ сходящихся п-тей, эл-ты которых стремятся к 0, $x=(x_1, x_2, \dots)$ с операциями сложения и умножения на число и нормой $\norm{x}_C = \sup\limits_{k \in N} |x_k|$.
\end{defn}

\begin{defn}
Пусть $X$ --- комплексное ЛП. \bf{Скалярным произведением} в $X$ наз-ся отображение $(\cdot, \cdot): X \cross X \ra \C$, т. ч.:
\begin{enumerate}
\i $\forall x \in X \Have (x, x) \in \R$ и $(x, x) \ge 0$;
\i $(x, x)=0 \Iff x = 0$;
\i $\forall x, y \in X \Have (x, y) = \bar{(y, x)}$;
\i $\forall x, y, z \in X \forall \alpha, \beta \in \C \Have (\alpha x + \beta y, z) = \alpha (x, z) + \beta(y, z)$.
\end{enumerate}
\end{defn}

\begin{defn}
ЛП с фиксированным в нём скалярным произведением наз-ся \bf{евклидовым}.
\end{defn}

\begin{stmt}
Пусть $X$ --- евклидово пр-во. Тогда величина $\norm{x} = \sqrt{(x, x)}, x \in X$, удовлетворяет определению нормы в $X$. Такая норма называется \bf{нормой, порождённой скалярным произведением}. 
\end{stmt}

\begin{defn}
ЕП, полное относительно нормы, порождённой скалярным произв., наз. \bf{гильбертовым пр-вом (ГП)} (обычно обозначается $H$).
\end{defn}

\begin{thm}[4.2]
Пусть $E$ --- ЛНП. Тогда норма в $E$ порождена скалярным произведением $\Iff$ выполняется \bf{равенство параллелограмма}: $\norm{x+y}^2+\norm{x-y}^2 = 2(\norm{x}^2+\norm{y}^2)$.
\end{thm}

\begin{thm}[4.3, Рисса о проекциях]
Пусть $H$ --- ГП, $M \subset H$ --- подпр-во. Тогда $H = M \oplus M^\perp$, где $M^\perp = \{y \mid (m, y) = 0 \forall m \in M\}$ --- \bf{аннулятор}.
\end{thm}

\begin{thm}[4.4]
Пусть $H$ --- ГП над $\R$ или $\C$, $e = \{e_n\}_{n=1}^\infty$ --- ОНС. Тогда СУЭ:
\begin{enumerate}
\i $e$ --- базис
\i $e$ --- полная система
\i $e^\perp = \{0\}$
\i $\forall x \in H$ справедливо равенство Парсеваля $\norm{x}^2 = \sum |(x, e_n)|^2$.
\end{enumerate}
\end{thm}

\begin{thm}[Рисса-Фишера]
Пусть $H$ --- ГП, $\{e_n\}$ --- ОНС. Тогда $\sum \alpha_n e_n$ сходится $\Iff$ $\sum |\alpha_n|^2$ сх-ся.
\end{thm}

\section{ЛОО}

\begin{defn}
Пусть $L$ --- действительное ЛП, и $x, y \in L$. Назовём \bf{замкнутым отрезком} в L, соединяющим точки $x$ и $y$, совокупность $\{\alpha x + \beta y \mid \alpha, \beta \ge 0, \alpha + \beta = 1\}$. Отрезок без концевых точек $x, y$ называется \bf{открытым отрезком}.
Мн-во $M \subset L$ наз-ся выпуклым, если оно вместе с любыми двумя точками $x$ и $y$ содержит соединяющий их отрезок.
\end{defn}

\begin{defn}
Пусть $X, Y$ --- ЛП. Линейное отображение $A: X \ra Y$ наз-ся \bf{линейным оператором}.
\end{defn}

\begin{defn}
Пусть $X, Y$ --- ЛП, $A: X \ra Y$ --- линейный оператор. \bf{Ядром} линейного оператора $A$ наз-ся подпр-во из $X$ вида $\Ker A = \{x \in X \mid Ax = 0\}$. \bf{Образом} (или \bf{мн-вом значений}) оператора $A$ наз-ся подпр-во из $Y$ вида $\Im A = \{Ax \mid x \in X\}$.
\end{defn}

\begin{defn}
Пусть $(X, \norm{\cdot}_X)$ и $(Y, \norm{\cdot}_Y)$ --- ЛНП. Лин. опер. $A: X \ra Y$ наз. \bf{ограниченным}, если $\forall$ ограниченного мн-ва $S \subset X$ его образ $A(S)$ является ограниченным в $Y$. 
\end{defn}

\begin{thm}[5.1]
Пусть $E_1, E_2$ --- ЛНП, $A: E_1 \ra E_2$: --- лин. опер. Тогда $A$ --- непр. $\Iff$ $A$ --- огр.
\end{thm}

%Где следующее утверждение было в лекциях?
%\begin{stmt}
%Пусть $(X, \norm{\cdot}_X)$ и $(Y, \norm{\cdot}_Y)$ --- ЛНП, $A: X %\ra Y$ --- лин. опер. Тогда СУЭ:
%\begin{enumerate}
%\i $A$ непрерывен в $X$
%\i $A$ непрерывен в нуле
%\i $A$ ограничен
%\i $\exists R > 0: A(B_1^X(0)) \subset B_R^Y(0)$, где $B_r^X (x) = %\{z \in X \mid \norm{z-x}_X \le r\}$.
%\end{enumerate}
%\end{stmt}

\begin{defn}
Пусть $(X, \norm{\cdot}_X)$ и $(Y, \norm{\cdot}_Y)$ --- ЛНП. \bf{Нормой} лин. опер. $A: X \ra Y$ наз-ся $\norm{A} = \inf\{k \mid \norm{Ax}_Y \le k\norm{x}_X \forall x \in X \}$.
\end{defn}

\begin{stmt}
Пусть $(X, \norm{\cdot}_X)$ и $(Y, \norm{\cdot}_Y)$ --- ЛНП, $A: X \ra Y$ --- лин. опер. Тогда 
$\norm{A} 
= \sup\limits_{x \ne 0} \frac{\norm{Ax}_Y}{\norm{x}_X} 
= \sup\limits_{\norm{x}_X = 1} \norm{Ax}_Y 
= \sup\limits_{\norm{x}_X \le 1} \norm{Ax}_Y$
\end{stmt}

\begin{stmt}
$\norm{A} < +\infty \Iff $ лин. опер. $A$ ограничен.
\end{stmt}

\begin{thm}[5.2]
Пусть $E_1, E_2$ --- ЛНП. Обозначим $L(E_1, E_2)$ --- мн-во лин. огр. операторов. Если на нём определить функции "$+$" и "$\cdot$", оно будет ЛП. Тогда:
\begin{enumerate}
\i Оно будет ЛНП, если $\norm{A}$ сделать нормой в $L(E_1, E_2)$
\i Оно будет БП, если $E_2$ --- БП.
\end{enumerate}
\end{thm}
\begin{cor}
$E$ --- БП $\Then$ $L(E)$ --- БП.
\end{cor}
\begin{cor}
$E$ --- НП $\Then$ $L(E, \R(\C))$ --- БП.
\end{cor}

\begin{thm}[5.3]
Пусть $E_1$ --- ЛНП, $E_2$ --- БП. $D(A) := \{$линейное многообразие в $E, D(A)$ всюду плотно в $E\}$, где $A$ --- лин. огр. оп., $A: D(A) \ra E_2$.

Тогда $\exists! \wave{A} \in L(E_1, E_2): \begin{cases} \wave{A}\mid_{D(A)} = A \\ \norm{\wave{A}} = \norm{A} \end{cases}$.
\end{thm}

\begin{thm}[5.4, Банах-Штейнгауз, "принцип равномерной ограниченности"]
Пусть $E_1$ --- БП, $E_2$ --- НП, $A_n \in L(E_1, E_2): \forall x \in E_1 \sup\limits_n \norm{A_n x} < \infty$.

Тогда $\sup\limits_n \norm{A_n} < \infty$
\end{thm}

\begin{thm}[сл-вие 5.4, полнота $L(E_1, E_2))$ в смысле поточечной сх-ти]
Пусть $E_1, E_2$ --- БП, $\{A_n\} \subset L(E_1, E_2)$, и $\forall x \in E_1 \{A_nx\}$ --- фунд.

Тогда $\exists A \in L(E_1, E_2): A_n \ra A$ поточечно.
\end{thm}

\begin{thm}[сл-вие 5.4, критерий поточ. сх-ти лин. огр. оператора]
Пусть $E_1, E_2$ --- БП, $\{A_n\} \subset L(E_1, E_2)$, и $ A \in L(E_1, E_2)$ --- лин. огр. опер. 

Тогда $A_n \to A$ поточ. $\Iff \begin{cases} \{\norm{A_n}\} \text{ огр.} \\ A_n x \ra Ax \forall x \in S: \bar{[S]} = E_1 \end{cases}$
\end{thm}

\begin{defn}
$L_1$ --- ЛНП функций с нормой $\norm{f}_{L_1} = \int |f(x)| dx$.

\bf{Полное лин. норм., но не евклидово}
\end{defn}

\section{Обр. опер.}

\begin{defn}
Пусть $E_1, E_2$ --- ЛП, $A: E_1 \ra E_2$. На $\Im A = \{y \in E_2\mid \exists$ решение $Ax=y$ в $E_1\}$ определён \bf{обратный оператор} $A^{-1}$, если $\forall y \in \Im A \exists! x \in E_1: Ax=y$.
\end{defn}

\begin{thm}[6.1]
Пусть $E_1, E_2$ --- ЛП, $A \in L(E_1, E_2)$. 

Тогда $\exists A^{-1} \in L(\Im A, E_1) \Iff \exists M>0: \norm{Ax} \ge m \norm{x} \forall x \in E_1$.
\end{thm}

\begin{thm}[6.2]
Пусть $E$ --- БП, $A \in L(E): \norm{A} < 1$. 

Тогда $\exists (I+A)^{-1} \in L(E)$.
\end{thm}

\begin{thm}[6.3]
Пусть $E_1$ --- БП, $E_2$ --- НП, $A \in L(E_1, E_2), \exists A^{-1} \in L(E_2, E_1), \Delta A \in L(E_1, E_2): \norm{\Delta A} < \norm{A^{-1}}^{-1}$.

Тогда $\exists (A+\Delta A)^{-1} \in L(E_2)$.
\end{thm}

\begin{thm}[6.4, Банаха об обратном операторе]
Пусть $E_1, E_2$ --- БП, $A \in L(E_1, E_2)$, $A$ --- биекция.

Тогда $\exists A^{-1} \in L(E_2, E_1)$.
\end{thm}

\section{Сопряжённое пространство}
Следующий семестр.

\end{document}
