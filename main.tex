\input{boilerplate.tex}

\date{}


\begin{document}
\bibliographystyle{unsrt}
%\bibliographystyle{gost780s}

\paragraph{Введение}
Пусть дан полный граф $K_n = (V, E)$ с весами рёбер $c \in \R^E$. Задача коммивояжёра заключается в том, чтобы найти гамильтонов цикл (обход) в G минимальной стоимости. Если $c_{ij} = c_{ji} \forall (i, j)  \in E$, говорят, что задача симметрична. 

Если веса рёбер удовлетворяют неравенству треугольника, говорят о метрической задаче коммивояжёра. Метрическая задача коммивояжёра --- одна из старейших известных NP-полных задач (доказательство NP-полноты см. в \cite{JP85}). Наилучший известный сейчас результат --- приближение с точностью $\frac{3}{2}$\cite{C76}. 

Внимания заслуживает случай симметричной задачи коммивояжёра с весами 1 и 2, называемый (1,2)-TSP. Поскольку $\forall a, b, c \in \{1, 2\} a\le b+c$, неравенство треугольника всегда выполняется, и это --- подкласс метрической задачи коммивояжёра. (1, 2)-TSP можно рассматривать как обобщение HAMPATH, требующее наличия в гамильтоновом пути как можно меньшего количества рёбер веса 2. Отсюда следует, что (1,2)-TSP NP-полно \cite{K72}. 

Было доказано \cite{EK01}, что нельзя построить приближение лучше $\frac{741}{740}$. Наилучший известный алгоритм Пападимитриу-Яннакакиса \cite{PY93} предоставлял приближение $7/6$ и был незначительно улучшен до $65/56$ \cite{BR05}. 

\paragraph{Эквивалентная формулировка}
Представим экземпляр задачи (1,2)-TSP как граф $G$, вершины которого --- точки метрики, а рёбра --- пары вершин на расстоянии 1. Пусть в $G$ $n$ вершин, и можно найти его покрытие $k$ путями (простыми и вершинно непересекающимися). Тогда эти пути содержат $n-k$ рёбер и их можно соединить в обход, где ребро из конца одного пути в начало другого будет иметь вес 2. Стоимость обхода будет $n+k$, задача --- минимизировать $k$. Если оптимальное решение содержит $n+k^*$ рёбер, нам нужно найти покрытие путями из $\le \frac{1}{7}n+\frac{8}{7}k^*$ путей. Будем далее решать такую задачу.

\paragraph{Алгоритм малых улучшений}
Будем поддерживать текущее решение алгоритма как множество рёбер $A$ ("algorithm"), являющееся 2-паросочетанием (то есть, у вершины может быть не более двух инцидентных ей рёбер из $A$). Пусть в нём $k_A$ путей и циклов, $m_A$ вершин в циклах, $s_A$ синглтонов (т. е. компонент связности размера 1). Это решение можно преобразовать с помощью множества вершин $C$ ("change") в новое решение $A \oplus C$. $C$ улучшает $A$, если 
\begin{enumerate}
	\i $A \oplus C$ --- 2-паросочетание;
	\i $k_{A \oplus C} < k_A$ или
	\i $k_{A \oplus C} = k_A$ и $m_{A \oplus C} > m_A$, или
	\i $k_{A \oplus C} = k_A$ и $m_{A \oplus C} = m_A$ и $s_{A \oplus C} < s_A$
\end{enumerate}

Алгоритм K-IMPROV:
$A = \emptyset$
while $\exists C, |C|<K, C$ улучшает $A$:
$A := A \oplus C$

Предположим, что для некоторой константы $K$ выполняется 
$$(*) \text{ либо } k_A \le \frac{1}{7}n+\frac{8}{7}k^*, \text{ либо } \exists C: C \text{ улучшает } A \textbf{ и } |C|\le K.$$
Нельзя провести более чем $n$ улучшений вида 2, т. к. число путей и циклов получится $\le0$.

Нельзя провести более чем $n$ последовательных подряд без улучшения вида 1, т. к. в циклах получится $>n$ вершин.

Нельзя провести более чем $n$ последовательных улучшений вида 3, т. к. иначе будет $<= 0$ изолированных вершин. 

То есть, можно провести не более чем $n^3$ улучшений. Поиск каждого улучшения занимает $O(n^K)$ шагов.

Очевидно, K-IMPROV работает за $O(n^{K+4})$. 

\paragraph{Доказательство утверждения (*)}

\bibliography{bibliography}
\end{document}
