\input{boilerplate.tex}

\date{}

\renewcommand{\G}{\mathcal{G}}
\begin{document}
\bibliographystyle{unsrt}
%\bibliographystyle{gost780s}

\paragraph{Введение}
Пусть дан полный граф $K_n = (V, E)$ с весами рёбер $c \in \R^E$. Задача коммивояжёра заключается в том, чтобы найти гамильтонов цикл (обход) в G минимальной стоимости. Если $c_{ij} = c_{ji} \forall (i, j)  \in E$, говорят, что задача симметрична. 

Если веса рёбер удовлетворяют неравенству треугольника, говорят о метрической задаче коммивояжёра. Метрическая задача коммивояжёра --- одна из старейших известных NP-полных задач (доказательство NP-полноты см. в \cite{JP85}). Наилучший известный сейчас результат --- приближение с точностью $\frac{3}{2}$\cite{C76}. 

Внимания заслуживает случай симметричной задачи коммивояжёра с весами 1 и 2, называемый (1,2)-TSP. Поскольку $\forall a, b, c \in \{1, 2\} a\le b+c$, неравенство треугольника всегда выполняется, и это --- подкласс метрической задачи коммивояжёра. (1, 2)-TSP можно рассматривать как обобщение HAMPATH, требующее наличия в гамильтоновом пути как можно меньшего количества рёбер веса 2. Отсюда следует, что (1,2)-TSP NP-полно \cite{K72}. 

Было доказано \cite{EK01}, что нельзя построить приближение лучше $\frac{741}{740}$. Наилучший известный алгоритм Пападимитриу-Яннакакиса \cite{PY93} предоставлял приближение $7/6$ и был незначительно улучшен до $65/56$ \cite{BR05}. 

\paragraph{Эквивалентная формулировка}
Представим экземпляр задачи (1,2)-TSP как граф $G$, вершины которого --- точки метрики, а рёбра --- пары вершин на расстоянии 1. Пусть в $G$ $n$ вершин, и можно найти его покрытие $k$ путями (простыми и вершинно непересекающимися). Тогда эти пути содержат $n-k$ рёбер и их можно соединить в обход, где ребро из конца одного пути в начало другого будет иметь вес 2. Стоимость обхода будет $n+k$, задача --- минимизировать $k$. Если оптимальное решение содержит $n+k^*$ рёбер, нам нужно найти покрытие путями из $\le \frac{1}{7}n+\frac{8}{7}k^*$ путей. Будем далее решать такую задачу.

\paragraph{Алгоритм малых улучшений}
Будем поддерживать текущее решение алгоритма как множество рёбер $A$ ("algorithm"), являющееся 2-паросочетанием (то есть, у вершины может быть не более двух инцидентных ей рёбер из $A$). Пусть в нём $k_A$ путей и циклов, $m_A$ вершин в циклах, $s_A$ синглтонов (т. е. компонент связности размера 1). Это решение можно преобразовать с помощью множества вершин $C$ ("change") в новое решение $A \oplus C$. $C$ улучшает $A$, если 
\begin{enumerate}
	\i $A \oplus C$ --- 2-паросочетание;
	\i $k_{A \oplus C} < k_A$ или
	\i $k_{A \oplus C} = k_A$ и $m_{A \oplus C} > m_A$, или
	\i $k_{A \oplus C} = k_A$ и $m_{A \oplus C} = m_A$ и $s_{A \oplus C} < s_A$
\end{enumerate}

Алгоритм K-IMPROV:
$A = \emptyset$
while $\exists C, |C|<K, C$ улучшает $A$:
$A := A \oplus C$

Предположим, что для некоторой константы $K$ выполняется 
$$(*) \text{ либо } k_A \le \frac{1}{7}n+\frac{8}{7}k^*, \text{ либо } \exists C: C \text{ улучшает } A \textbf{ и } |C|\le K.$$
Нельзя провести более чем $n$ улучшений вида 2, т. к. число путей и циклов получится $\le0$.

Нельзя провести более чем $n$ последовательных подряд без улучшения вида 1, т. к. в циклах получится $>n$ вершин.

Нельзя провести более чем $n$ последовательных улучшений вида 3, т. к. иначе будет $<= 0$ изолированных вершин. 

То есть, можно провести не более чем $n^3$ улучшений. Поиск каждого улучшения занимает $O(n^K)$ шагов.

Очевидно, K-IMPROV работает за $O(n^{K+4})$. 

\paragraph{Доказательство утверждения (*)}
Зафиксируем оптимальное решение, 2-паросочетание $B$ ("best"), т. ч. $k_B = k^*$. Пусть $D$ --- множество рёбер, оба конца которых лежат в одном и том же цикле $A$. Пусть $\G$, \textit{вспомогательный граф}, --- граф, содержащий вершины графа $G$ и рёбра $(A \cup B) \setminus D$. Покрасим его рёбра в три цвета.
Белые $A \setminus B \setminus D$, чёрные $B \setminus A \setminus D$ и серые $(A \cap B) \setminus D$. 

Назовём \textit{чередующимся путём} (ЧП) путь, начинающийся и заканчивающийся в чёрной вершине, в котором чёрные и белые вершины чередуются.

Пути и циклы $A$ назовём A-объектами. Для A-объекта определим \textit{начальную вершину} как вершину, которая может быть первой или последней вершиной ЧП и при этом \textit{принадлежит} A-объекту. Для A-пути, начальные вершины --- его концы. Для A-цикла $C$ возьмём за начальные вершины такие две, что в $C$ существует гамильтонов путь из одной в другую, который может быть продолжен двумя чёрными рёбрами до двух других вершин. 

Покажем, что такая пара всегда найдётся в A-цикле из менее 8 вершин.

Введём понятие \textit{фишки обоснования}. Это абистрактная делимая единица, которая ассоциирована с вершинами. Изначальное распределение фишек задаётся весами вершин. Общее количество фишек --- сумма этих весов.
Если $B$ состоит из $k^*$ путей, оптимум имеет стоимость $n+k^*$ и A даёт достаточное приближение, если она имеет стоимость $\le \frac{8}{7}(n+k^*)$, т. е. если она состоит из $\le \frac{1}{7} + \frac{8}{7}k^*$ A-объектов. Создадим $n+8k^*$ фишек и докажем, что для того, чтобы A было достаточно хорошим приближением, каждый A-объект должен содержать 8 точек.
Вершина, инцидентная $2-a$ вершинам B, содержит $1+4а$ фишек (сумма их всех $2k^*$). Изначально фишки пути есть фишки его концов и фишки цикла есть фишки всех его вершин. Оставшиеся фишки содержатся в ЧП. ЧП даёт фишки тем A-объектам, которые содержат его начальные вершины. После "разбиения" ЧП, они могут содержать только одну начальную вершину, и поэтому дают свои фишки только одному A-объекту.

Обычно, А-путь содержит два конца, и каждый из них --- начальная вершина ЧП, который даёт этому пути $2
frac{1}{2}$ фишек. Похожим образом, цикл обычно содержит $4+a$ вершин и содержит 2 начальных верин ЧП, которые дают циклу $\frac{3-a}{2}$ вершин. 

Могут быть отклонения от обычного случая. Если А-объект содержит менее 2-х начальных вершин, он собирает больше фишек с вершин, которые он содержит. Если А-объект --- синглтон, то каждая начальная вершина даёт ему 3 фишки. Если цикл содержит более 6 вершин, в нём не будет начальных вершин. Вершина, инцидентная только одному ребру из $B$ имеет 4 дополнительных фишки, и мы будем опускать особые случаи, вызванные такими вершинами.

\paragraph{Очень маленькие улучшения}
В некоторых случаях возможны улучшения, вставляющие только одно ребро. Обсудим их, а дальше будем предполагать, что они не встречаются.

Чёрное ребро $e$, соединяющее две начальные вершины. Если оно соединяет два разных А-обхекта, они сливаются в один. Если сливаемый объект-цикл, приходится удалить одно из его рёбер. Если $e$ содержит начальные вершины одного А-объекта, это обязан быть А-путь, и вставка ребра превращает этот путь в цикл.

Ребро, соединяющее А-синглтон с другим А-объектом (за исключением случая, когда синглтон соединяется со средней вершиной А-пути из трёх вершин -- иначе улучшение типа 4).

Предположим, что нет очень маленького улучшения, и есть ЧП $R$, начинающийся в $u$, а $\{u\}$ --- А-синглтон. Тогда для пути (v, w, x) и некоторого $y$ путь $R$ начинается с $(u, w, v, y)$. Когда мы считаем $R$ возможной частью улучшения, мы берём "сокращённую" версию $R$, начинающуюся с $(v, y)$. С одной стороны, мы забываем, что $R$ начинается с синглтона и поэтому должен собрать лишнюю $\frac{1}{2}$ фишки. С другой стороны, мы забудем, что $R$ собирает $\frac{1}{2}$ фишки на вершине $w$.

\paragraph{Примеры с ЧП}
ЧП сам по себе может задавать улучшение. 

\paragraph{Начальные вершины циклов}
Пусть $C$ --- цикл A с $\ge 7$ вершинами с $|C|$ фишками (т. е., у него все вершины смежны с двумя рёбрами из $B$).

Пусть $\hat{C}$ --- множество вершин из $C$, и $\hat{K} \subset \hat{C}$ --- множество вершин, инцидентных двум чёрным вершинам. 

Покажем, что какие-то две вершины из $\hat{K}$ \textit{совместимы} в том смысле, что они --- концы гамильтонова пути в $\hat{C}$.

Если $|\hat{K}| = 2$, то $\hat{K}$ --- \textit{согласованная пара}, потому что множество рёбер $B$, содержащихся в $C$, образует единственный путь.

Пусть $|\hat{K}| \ge 3$. 
Предположим, что две вершины $\hat{K}$, $u$ и $v$, соседние на цикле $C$. Тогда они согласованы, т. к. мы можем сделать путь, удалив ребро $(u, v)$ из $C$.
Иначе они не смежны, и получаем, что $|\hat{K}| \le |\hat{C}|/2$. Это означает, что $|\hat{K}| = 3$ и $|\hat{C}| = 6$.

Т. к. $\hat{K}$ содержит 3 вершины, $B$ покрывает $\hat{C}$ двумя путями. Один из этих путей содержит 1 вершину и 0 рёбер, другой 5 вершин и 4 ребра. Следовательно, 2 ребра $B$ содержатся в $\hat{C} \setminus \hat{K}$. БОО, $C$ --- цикл $(u_0, \dots, u_5)$, $\hat{K} = \{u_-0, u_2, u_4\}$ и $\{u_1, u_3\} ]in B$. Тогда мы можем обойти $\hat{C}$ так: $(u_0, u_5, u_4, u_3, u_1, u_2)$.

\paragraph{ЧП с дефицитом --- общий метод}
Пусть $R$ - ЧП. По нашим правилам, он собирает $\frac{1}{2}$ фишек за каждую свою вершину, за исключением концов путей и вершин циклов. Мы не всегда собираем больше потому, что каждая из таких вершин может принадлежать больше чем одному ЧП.

Есть несколько методов дать $R$ больше фишек.
Распределить фишки серых вершин.
Разбить ЧП $P$, проходящий через цикл, созданный $R$. Если он короткий, можно слить цикл с A-объектом, содержащим начальную точку $P$. Если он длинный, ему не надо собирать фишки с его рёбер в цикле, и эти фишки можно передать $R$.

\paragraph{Избегаем плохих случаев}
\subparagraph{S-дуги --- Избегаем их или находим им дополнительные фишки}

\begin{defn} \end{defn}

\bibliography{bibliography}
\end{document}
