\documentclass[a4paper,10pt, russian]{article}
\usepackage[T1, T2A]{fontenc}
\usepackage[utf8]{inputenc}
\usepackage[english,russian]{babel}
\usepackage{amsgen, amsmath, amstext, amsbsy, amsopn, amsfonts, amsthm, thmtools,  amssymb, amscd, mathtext}
\usepackage{versions}
\usepackage[table]{xcolor}
%\usepackage{upgreek}
\usepackage{dsfont}
\usepackage{graphicx}
\usepackage{textcomp}
\usepackage{geometry}
  \geometry{left=2cm}
  \geometry{right=1.5cm}
  \geometry{top=1cm}
  \geometry{bottom=2cm}
  
\usepackage{tikz}
\usepackage{ccaption}
\usepackage{wrapfig}

\usepackage{minted}
\newcommand{\ci}{\mintinline{c++}}

\usepackage{hyperref}
\definecolor{linkcolor}{HTML}{799B03}
\definecolor{urlcolor}{HTML}{0000FF}
\hypersetup{pdfstartview=FitH, linkcolor=linkcolor, urlcolor=urlcolor, colorlinks=true}

\declaretheoremstyle[notefont=\bfseries,notebraces={}{},headpunct={.},postheadspace=1em]{myproblemstyle}
\declaretheorem[style=myproblemstyle,numbered=no,name=Задача]{problem}

\declaretheoremstyle[notefont=\bfseries,notebraces={}{},headpunct={:},postheadspace=1em]{myproofstyle}
\declaretheorem[style=myproofstyle,numbered=no,name=Решение]{solution}

\declaretheoremstyle[notefont=\bfseries,notebraces={}{},headpunct={:},postheadspace=1em]{myanswerstyle}
\declaretheorem[style=myanswerstyle,numbered=no,name=\it{Ответ}]{answer}

\declaretheoremstyle[spacebelow=5pt,notefont=\bfseries,notebraces={}{},headpunct={.\\},postheadspace=1em]{mytaskstyle}
\declaretheorem[style=mytaskstyle,numbered=no,name=Задача]{task}

%\excludeversion{solution}
%\excludeversion{answer}

\newcommand{\dividedby}{\raisebox{-2pt}{\vdots}}

\title{Практикум №1 по алгоритмам}

\begin{document}
	\maketitle
	
	\section{Построение автомата}
	
	Заводится стек "подавтоматов" --- участков автомата, имеющих начало и конец и составляющих в будущем искомый автомат. Они хранятся в стеке.
	
	При поступлении на вход оператора из стека извлекается необходимое количество операндов, и из них строится новый подавтомат, который кладётся на вершину стека. Построение нового автомата осуществляется так же, как было на семинаре. 
	
	Подробнее --- в коде.
	
	\section{Поиск ответа}
	Будем искать длину максимального слова, состоящего из x. Если она больше или равна k, то нужное слово есть, иначе нет.
	
	Будем ходить обходом в глубину по автомату. Если из вершины выходит ребро с меткой x, то двигаемся по нему, увеличивая глубину на 1. Если $\epsilon$-ребро, то оставляем глубину такой же. Рёбра с другими пометками нас не интересуют, так как они обрывают цепочку из переходов по x, то есть обрывают искомое слово. В процессе поиска в глубину мы можем попадать в белые, серые и чёрные вершины. 
	
	Пусть мы попали в $v$. $v$ является чёрной, если dfs из неё уже отработал, найдя максимальную длину слова. Просто возвращаем ответ. Если вершина серая, то мы в процессе обхода нашли цикл. Если при этом глубина $depth[v]$ больше, чем текущая глубина обхода, то мы нашли цикл из букв x и, возможно, некоторого количества $\epsilon$-переходов. По нему мы можем получить произвольную длину слова. Если же глубина такая же, это значит, что цикл весь состоял из $\epsilon$-переходов. Он нас не интересует. Если же вершина белая, то
	необходимо запустить из неё dfs (причём если мы пришли в неё по x, то увеличиваем глубину и ответ на 1, если по $\epsilon$ --- оставляем прежними; подробнее в коде). Улучшаем ответ, если можно.
	
	Запустив так dfs из всех вершин, мы найдём максимальную длину слова.
	Если мы найдём цикл по $x$ и $\epsilon$, то, очевидно, алгоритм корректен (а dfs всегда находит цикл по факту, доказанному на курсе алгоритмов). Иначе же слово $x^k$ имеет вид цепочки из переходов $x$. Эта цепочка где-то начинается. Если мы запустим dfs из этой вершины, то найдём такую цепочку (очевидно). Поэтому алгоритм корректен.
	
	\section{Время работы}
	Пусть $N$ --- длина регулярного выражения. Построение автомата выполняется за один проход, один символ обрабатывается за $O(1)$. dfs работает за $O(N)$, запускается $N$ раз, итого $O(N^2)$. Восстановление ответа за $O(N)$. Общее время работы $O(N^2)$.
	
	Вообще, в целях оптимизации можно заканчивать алгоритм, как только было найдено слово нужной длины, но особого значения это не имеет.
	
	\section{Проверка регулярного выражения на корректность}
	Единственная проблема, с которой можно столкнуться --- несоответствие количества операторов и нужных операндов. 
	
	Если считали оператор, проверяем, находится ли в стеке нужное количество операндов (равное арности оператора), и если не так, бросаем исключение.
	
	Если после окончания работы в стеке больше чем один подавтомат (меньше уж точно быть не может, так как оператор всегда кладёт подавтомат в стек, а извлечь больше, чем в стеке есть элементов, нельзя), у нас недостаточно операторов, и следует бросить исключение.

\end{document}
